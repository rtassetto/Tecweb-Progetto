\section{Abstract}
Il progetto si propone di realizzare il sito web di un rivenditore di componenti per PC.
Lo scopo principale è quello di fornire ai clienti una interfaccia dalla quale possono visualizzare e ordinare i pezzi che vogliono acquistare, il sito quindi offre la classica funzionalità di \emph{carrello}, dove l'utente può inserire i componenti che vorrebbe acquistare, inoltre per ogni utente è disponibile la storia degli acquisti effettuati, da questa lista l'utente è in grado di dare una valutazione e di scrivere una recensione sul prodotto.
Un admin può aggiungere e rimuovere oggetti dal catalogo, e creare dei Bundle che verranno visualizzati nella Home.
\section{Utenti destinatari}
I clienti del negozio sono generalmente persone già informate su cosa vogliono acquistare e sulla funzione dei vari componenti, perciò vengono fornite le caratteristiche tecniche di ogni pezzo, gli utenti possono inoltre fare ricerche mirate sul nome di un componente oppure tramite una ricerca avanzata possono selezionare un tipo specifico di componenti (RAM,Processore,ecc...).
Per non lasciare fuori gli utenti meno esperti forniamo nella home page una serie di computer preassemblati su diverse fasce di prezzo con una semplice descrizione che aiuti i clienti meno esperti a capire se una data macchina soddisfa o meno i suoi bisogni.
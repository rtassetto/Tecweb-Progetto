\subsection{Presentazione}

Nel realizzare l'interfaccia grafica del sito è stato usato lo standard CSS3.\newline
I diversi file .css utilizzati ci permettono di realizzare una formattazione adatta per la visione del sito da un desktop, un mobile e per un layout di stampa.\newline
I file utilizzati sono rispettivamente:
\begin{itemize}
\item\textit{style.css};
\item\textit{small.css};
\item\textit{print.css}.
\end{itemize}
Grazie alla funzionalità che ci mette a disposizione questa versione dello standard, ovvero le \textit{media query}, siamo in grado di definire dei "breakpoint" e quindi di modellare lo stile di visualizzazione in base al device utilizzato, facendo riferimento direttamente alle dimensioni dello schermo.\newline
Per la visualizzazione da desktop abbiamo deciso di visualizzare un menù orizzontale affiancato dal logo del sito e da una barra di ricerca, sfruttando la piena larghezza dello schermo.\newline 
Mentre per la visualizzazione da mobile abbiamo scelto di mostrare il menù in verticale, di non visualizzare il logo del sito e di tenere la barra di ricerca sotto al menù, in modo che sia subito visibile all'utente.\newline
\newline
Abbiamo inoltre scelto di non utilizzare la stessa formattazione tra le pagine viste dagli utenti e quelle viste dagli amministratori. Infatti nelle pagine viste dagli utenti, come ad esempio \textit{prodotti.php} che racchiude il catalogo dei prodotti disponibili nel nostro sito, abbiamo scelto un layout costituito da più blocchi \textit{div} ognuno dei quali contiene un singolo prodotto, accompagnato dall'immagine e da tutte le informazioni utili all'acquisto.\newline
In questo modo crediamo che l'utente possa trovare minore difficoltà nel cercare le informazioni di cui ha bisogno, piuttosto che trovarsi davanti ad una tabella, ricca di dati, ma di difficile interpretazione.\newline
Diversamente, nella parte del sito dedicata agli amministratori, abbiamo pensato appunto a strutture tabellari (esempio \textit{adminproducts.php}), presupponendo che chi amministra il sito sia in grado di muoversi all'interno della sezione e rendendo la tabella più semplice e chiara possibile.
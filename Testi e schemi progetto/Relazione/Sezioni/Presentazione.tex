\subsection{Presentazione}
Nel realizzare l'interfaccia grafica del sito è stato usato lo standard CSS3.\newline
I diversi file .css utilizzati ci permettono di realizzare una formattazione adatta per la visione del sito da un desktop, un mobile e per un layout di stampa, nelle seguenti sezioni descriveremo le scelte adottate per la presentazione CSS di ogni pagina.

\subsubsection{style.css}

\paragraph{Header} \mbox{}
l'header è una parte comune a tutte le pagine del sito e contiene il menù di navigazione, per la sua presentazione abbiamo utilizzato il comando \emph{float:left} per tenere a sinistra il logo all'interno del div principale, inoltre gli è stato data una \emph{max-width} di 10em per limitare la sua dimensione. all' UL di classe \emph{nodot} è stato dato il comando \emph{list-style-type:none} per rimuovere i punti dell'elenco e i comandi \emph{margin:0} e \emph{padding:0} per mantenere il menù della dimensione corretta, all'ultimo tag \emph{li} che contiene la barra di ricerca è stato dato il comando \emph{float:right} per ottenere una divisione spaziale tra i link di navigazione e la ricerca.
A tutti i tag \emph{li} del menù è stato dato il comando \emph{display:block} per essere sicuri che ogni voce abbia il proprio blocco e per allinearli abbiamo usato il comando \emph{float:left}.
Infine al div principale è stato dato il comando \emph{width:100\%} per ottenere un menù di navigazione che occupa sempre tutto lo schermo in larghezza.


\paragraph{Home} \mbox{}
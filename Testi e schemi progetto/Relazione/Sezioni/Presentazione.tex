\subsection{Presentazione}
Nel realizzare l'interfaccia grafica del sito è stato usato lo standard CSS3.\newline
I diversi file .css utilizzati ci permettono di realizzare una formattazione adatta per la visione del sito da un desktop, un mobile e per un layout di stampa, nelle seguenti sezioni descriveremo le scelte adottate per la presentazione CSS di ogni pagina.

\subsubsection{style.css}
in questa sezione vengono descritti i comandi css usati per la presentazione del sito in formato desktop.

\paragraph{Header} \mbox{}
L'header è una parte comune a tutte le pagine del sito e contiene il menù di navigazione. \newline Per la sua presentazione abbiamo utilizzato il comando \emph{float:left} per tenere a sinistra il logo all'interno del div principale, inoltre gli è stato data una \emph{max-width} di 10em per limitare la sua dimensione. all' UL di classe \emph{nodot} è stato dato il comando \emph{list-style-type:none} per rimuovere i punti dell'elenco e i comandi \emph{margin:0} e \emph{padding:0} per mantenere il menù della dimensione corretta, all'ultimo tag \emph{li} che contiene la barra di ricerca è stato dato il comando \emph{float:right} per ottenere una divisione spaziale tra i link di navigazione e la ricerca.
A tutti i tag \emph{li} del menù è stato dato il comando \emph{display:block} per essere sicuri che ogni voce abbia il proprio blocco e per allinearli abbiamo usato il comando \emph{float:left}.
Infine al div principale è stato dato il comando \emph{width:100\%} per ottenere un menù di navigazione che occupa sempre tutto lo schermo in larghezza.

\paragraph{Footer} \mbox{}



\paragraph{Home} \mbox{}
La presentazione della home è stata fatta affiancando due \emph{section}, ovvero dando alla sezione di sinistra il comando \emph{float:left} e \emph{width 49\%} mentre alla sezione di destra sono stati dati i comandi \emph{margin:51\%} e \emph{width 49\%}. 
L'immagine dei prodotti è stata posizionata usando i comandi \emph{float:right} e \emph{margin-right:30\%} per posizionarla alla destra del testo che descrive il prodotto, inoltre è stato dato alla classe \emph{prodottobestseller} il comando overflow: auto in modo che il div contenga correttamente l'immagine e il testo.
Il comando \emph{overflow: auto} è stato dato anche al div di id \emph{content} per evitare sovrapposizioni con il footer durante il ridimensionamento della pagina.

\paragraph{Prodotti} \mbox{}
La pagina è presentata come una lista di \emph{div} all'interno di un tag \emph{section}.\newline All'interno di ogni \emph{div} appartenente alla classe \emph{prodotto}, troviamo oltre al nome del prodotto due div, uno appartenente alla classe \emph{info} (spostato a sinistra con un \emph{float:left} e una \emph{width:65\%}) e l'altro appartenente alla classe \emph{img} (affiancato alla destra del primo con un \emph{margin-left:65\%} e una \emph{width:35\%}). 


\subsubsection{small.css}
in questa sezione vengono descritti i comandi css usati per la presentazione del sito in formato mobile.

\paragraph{Generale} \mbox{}
Inanzi tutto è stato dato \emph{margin 0} e \emph{padding: 0} ad HTML e BODY per azzerare eventuali posizionamenti non desiderati. 

\paragraph{Header} \mbox{}
Inanzi tutto viene dato all'id \emph{menu} il comando \emph{display:none} per non visualizzare i link di navigazione destinati al formato desktop, inoltre viene visualizzata l'icona del menu ottenuta dando ai 3 div vuoti a i comandi \emph{border:solid}, \emph{width: 2em} e \emph{margin: 6px 0},questi tre div vuoti sono contenuti in un div di classe \emph{comp}, a questa classe viene dato il comando \emph{float: right} perchè rimanga sulla destra della barra superiore e le sono stati dati dei margini superiori e di destra per distanziare l'icona dai bordi.

\paragraph{Footer} \mbox{}
Nella versione mobile il footer contiene il menù di navigazione, per questo al l'intero div che ha \emph{mobile} come id viene dato il comando \emph{display: block} per sovrascrivere il comando \emph{display: none} dato in style.css inoltre vengondo dati dei margini per ottenere distanza dal resto della pagina.
Al tag \emph{ul} contenuto in nel menù viene data una width del 100\% e ai tag \emph{li} annidati che hanno classe \emph{voci} viene dato il comando \emph{float: none} per sovrascrivere il comando float dato in style.css, e il comando \emph{text-align: center} per centrare le voci, in oltre viengono impostati margini e padding laterali a 0.

\paragraph{Home} \mbox{}
Le due sezioni principali vengono messe una sotto l'altra utilizzando il comando \emph{float:none} sulla sezione con id Latestbundles per sovrascrivere il float dato in style.css



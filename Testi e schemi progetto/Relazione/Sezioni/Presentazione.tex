\subsection{Presentazione}
Nel realizzare l'interfaccia grafica del sito è stato usato lo standard CSS3.\newline
I diversi file .css utilizzati ci permettono di realizzare una formattazione adatta per la visione del sito da un desktop, da un dispositivo mobile e per un layout di stampa. \newline
Il sito presenta un layout responsive e attraverso une media query se il dispositivo di cui si sta facendo utilizzo ha uno schermo con larghezza minore o uguale a 480px, allora viene caricato il layout mobile.\newline
Altra scelta è stata quella di fissare, per il layout desktop, la larghezza della pagina a 1024px in modo da permettere una visione ordinata anche su schermi molto grandi.\newline
Nelle seguenti sezioni descriveremo le scelte adottate per la presentazione CSS di ogni pagina.

\subsubsection{style.css}
In questa sezione viene descritta la presentazione del sito in formato desktop.

\paragraph{Header} \Spazio
L'header è una parte comune a tutte le pagine del sito e contiene il logo del sito, il menù di navigazione e la barra di ricerca.\newline
Nella versione desktop il menù è rappresentato come un unico blocco nella parte superiore della pagina, con questi tre elementi allineati.

\paragraph{Footer} \Spazio
Per il footer, comune a tutte le pagine, si è utilizzato uno sfondo dello stesso colore del menù... 


\paragraph{Home} \Spazio
La presentazione della home è stata fatta affiancando due \emph{section}, contenenti rispettivamente i Bundle disponibili e i prodotti più venduti, in modo da sfruttare a pieno la larghezza della pagina.\newline 


\paragraph{Prodotti} \Spazio
La pagina è presentata come una lista di tutti i prodotti presenti nel sito.\newline
Inoltre possiamo trovare anche un form per la ricerca avanzata nella parte superiore della pagina. 


\subsubsection{small.css}
In questa sezione viene descritta la presentazione del sito in formato mobile.

\paragraph{Generale} \Spazio
Innanzitutto è stato dato \emph{margin 0} e \emph{padding: 0} ad HTML e BODY per azzerare eventuali posizionamenti non desiderati. 

\paragraph{Header} \Spazio
Diversamente dalla versione desktop, qui manteniamo solo il logo nella parte superiore e inseriamo un div cliccabile che fa da ancora al menù, posizionato ora in verticale nella parte inferiore del sito, seguito dalla barra di ricerca.\newline


\paragraph{Home} \Spazio
In questo caso, le due \emph{section} (prima allineate in orizzontale), vengono allineate verticalmente in modo da rendere la pagina più leggibile da mobile.


\subsubsection{print.css}
Per quanto riguarda il layout di stampa, abbiamo tolto il link del sito, il menù, le immagini e i vari sfondi e bordi presenti, in modo da lasciare solamente le informazioni necessarie all'utente.

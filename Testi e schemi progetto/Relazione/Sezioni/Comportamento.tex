\subsection{Comportamento}
Per definire il comportamento del sito è stato utilizzato PHP 7 e in qualche caso JavaScript. \newline
Per memorizzare i dati è stato utilizzato un database relazionale MySql.
\subsubsection{PHP}
PHP è stato utilizzato per la gestione del sito lato server, è stato utile per: 
\begin{itemize}
    \item la visualizzazione dinamica dei dati memorizzati nel database;
    \item la gestione degli utenti che effettuano il \emph{login};
    \item l'inserimento e il controllo di input di utenti all'interno del database;
    \item l'inserimento di dati nel database da parte degli admin.
\end{itemize}
Abbiamo raggruppato le parti ricorrenti in ogni pagina in dei file php contenuti nella cartella general chiamati:
\begin{itemize}
	\item \textit{Header.php;}
	\item \textit{Meta.php;}
	\item \textit{Footer.php.}
\end{itemize}
questi file vengono poi richiamati nelle varie pagine attraverso il comando \emph{require}.\newline

\paragraph{Connessione al database} \Spazio
La gestione della connessione al database è affidata alla classe \emph{DBAccess} contenuta in \textit{connessione.php}, questa classe oltre ad aprire la connessione al database contiene diverse funzioni che si occupano di interrogare con linguaggio SQL il database:
\begin{itemize}
	\item \textbf{createUser:} Serve per la registrazione degli utenti, controlla che i valori inseriti abbiano una definita lunghezza minima e massima, che il nome utente non sia già registrato e inserisce nel database il nuovo utente;
	\item \textbf{checkUser:} Se l'utente inserito è registrato permette la login,controlla anche se l'utente è un admin o meno;
	\item \textbf{alterAdminright:} toglie i privilegi di amministratore ad un admin;
	\item \textbf{getUserlist:} ritorna una array associativo contenente i dati di tutti gli utenti;
	\item \textbf{getP:} ritorna un array associativo contenente i dati di tutti i prodotti nel database;
	\item \textbf{ricerca:} questa funzione ritorna un array associativo contenente tutti i prodotti che in almeno uno dei campi nome,categoria o descrizione contiene una stringa data; 
	\item \textbf{ricercaAvanzata:} ritorna un array associativo contenente i dati di tutti i prodotti di uno specifico tipo in ordine di prezzo;
	\item \textbf{createProduct:} inserisce nel database un nuovo prodotto con i parametri dati;
	\item \textbf{getProddata:} ritorna una array associativo contenente i dati di uno specifico prodotto;
	\item \textbf{modifyProduct:} permette di modificare le informazioni riguardanti uno specifico prodotto contenuto nel database;
	\item \textbf{getBestselling:} ritorna un array associativo contenente i 6 prodotti più venduti; 
	\item \textbf{getfullPH:} ritorna un array associativo contenente tutto lo storico dei prodotti venduti;
	\item \textbf{getPH:} ritorna un array associativo contenente lo storico degli acquisti di uno specifico utente;
	\item \textbf{acquista:} inserisce nello storico degli acquisti i prodotti presenti nel carrello di uno specifico utente;
	\item \textbf{eliminaC:} se nel carrello di un dato utente vi è un dato prodotto con quantità 1 questo viene rimosso dal carrello, se la quantità è maggiore di 1 invece la quantità viene diminuita di 1;
	\item \textbf{aggiungiC:} aggiunge un prodotto al carrello di un dato utente, se il prodotto è già presente viene incrementata la quantità;
    \item \textbf{getCarrello:} ritorna un array associativo contenente i prodotti all'interno del carrello di uno specifico utente;
	\item \textbf{getProdReview:} ritorna un array associativo contenente tutte le recensioni di un dato prodotto;
	\item \textbf{aggiungiR:} inserisce nel database una recensione fatta da un utente per un dato prodotto;
	\item \textbf{getLatestBundles:} ritorna un array associativo contenente gli ultimi bundle creati; 
	\item \textbf{getB:} ritorna un array associativo contente i nomi di tutti i bundle;
	\item \textbf{getPB:} ritorna un array associativo contenente l'id di tutti i prodotti di un dato bundle;
	\item \textbf{aggiungiPB:} inserisce un prodotto in un dato bundle;
	\item \textbf{rimuoviPB:} rimuove un prodotto da un dato bundle;
	\item \textbf{creaB:} crea un nuovo bundle;
	\item \textbf{getBundlepartsdata:} ritorna un array associativo contenente le informazioni di tutti i prodotti contenuti in un dato bundle;
\end{itemize}

\paragraph{Controllo degli input} \Spazio
Le funzioni che si occupano di controllare gli input dati dagli utilizzatori del sito sono raggruppate nel file \textit{controlli.php},le funzioni presenti sono:

\begin{itemize}
    \item \textbf{desc():} questa funzione controlla che il testo inserito come recensione sia conforme alle attese. Grazie all'utilizzo di una espressione regolare vengono accettati tutti i testi con punteggiatura la quale deve essere per forza preceduta da un testo.\newline
    Questa funzione utilizza il passaggio di parametri per riferimento per cui avrà side effect sulle variabili utilizzate come parametri;
    
    
    \item \textbf{sostituzione():} questa funzione per prima cosa utilizza la funzione php \emph{addslashes()} che restituisce una stringa con il carattere di backslash '\textbackslash' anteposto ai caratteri che richiedono il quoting nelle query dei database.
    Successivamente vengono sostituite le lettere accentate del testo con le loro codifiche utf-8 per facilitare la lettura del testo anche all'interno del database.
    La funzione utilizza il passaggio dei parametri per riferimento per ottenere dei side effect sul parametro in entrata;
    
    \item \textbf{prezzo():} questa funzione controlla che la stringa in entrata sia un possibile prezzo utilizzando un'espressione regolare;
    
    \item \textbf{nomeProdotto():} questa funzione controlla che la stringa in entrata possa essere il nome di un prodotto utilizzando un'espressione regolare;
    
    \item \textbf{descProdotto():} questa funzione controlla che la stringa in entrata sia conforme ad essere la descrizione di un prodotto utilizzando un'espressione regolare;
    
    \item \textbf{nomeBundle():} questa funzione controlla che la stringa in entrata sia conforme ad essere il nome di un bundle attraverso un'espressione regolare;
    
    \item \textbf{checkLength():} questa funzione controlla che la stringa passata sia di lunghezza compresa tra 6 e 20 caratteri,
    se non è così ritorna il valore booleano true che verrà inserito in una variabile che segnala l'errore.
    Questa funzione viene utilizzata per il controllo degli input dati dall'utente nella pagina di registrazione;
    
    \item \textbf{checkUsername():} questa funzione controlla che la stringa inserita sia conforme ad essere utilizzata come username di un utente attraverso l'utilizzo di una espressione regolare;

   \item \textbf{checkPassword():} questa funzione controlla che la stringa passata come parametro sia conforme ad essere utilizzata come password di un utente attraverso l'utilizzo di una espressione regolare.

\end{itemize}

\paragraph{Gestione utente} \Spazio
Per gestire gli utenti abbiamo utilizzato la variabile \$\_SESSION\{login\_user\} nella quale viene mantenuto l'username degli utenti che effettuano il login fino a quando questi chiudono il browser, grazie a questa variabile l'username dei vari utenti è noto in ogni pagina e ci è dunque possibile mostrare a ognuno il proprio carrello e le informazioni sul proprio account come lo storico degli acquisti;
inoltre l'utilizzo di questa variabile ha reso semplice il logout che viene effettuato attraverso i comandi \emph{session\_unset} e \emph{session\_destroy}.

\subsubsection{Javascript}
Javascript è stato utilizzato per controllare a lato client se gli input degli utenti sono corretti o meno e per notificare all'utente la riuscita di operazioni come 
\emph{Acquista}, \emph{Aggiungi al carrello} e \emph{Inserimento recensioni}.

\subsubsection{Database MySQL}
I dati relativi al nostro sito sono stati salvati in un database relazionale SQL, queste sono le tabelle che costituiscono il database:
\begin{itemize}
    \item \textbf{Account:} username,password,email,admin,datacreazione;
    \item \textbf{Prodotto:} id,nome,categoria,descrizione,valutazione,prezzo;
    \item \textbf{PurchaseHistory:} idordine,compratore,prodotto,data,quantità;
    \item \textbf{Carrello:} username,prodotto,quantità;
    \item \textbf{Recensione:} username,prodotto,review,voto,data;
    \item \textbf{Bundles} nome,descrizione,data;
    \item \textbf{Bundleparts:} bundle,pezzo.	
\end{itemize}
nel database sono inoltre presenti 2 \emph{Trigger}:
\begin{itemize}
	\item \textbf{AggiornaVoto:} questo \emph{Trigger} aggiorna il voto di un prodotto dopo che un utente inserisce la propria valutazione su di esso;
	\item \textbf{AggiornaVotoEdit:} questo \emph{Trigger} aggiorna il voto di un prodotto dopo che un utente modifica la propria valutazione su di esso.
\end{itemize}
